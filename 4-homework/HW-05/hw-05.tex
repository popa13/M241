\documentclass[addpoints, 12pt]{exam}%, answers]
\usepackage[utf8]{inputenc}
\usepackage[T1]{fontenc}

\usepackage{lmodern}
\usepackage{arydshln}
\usepackage[margin=2cm]{geometry}

\usepackage{enumitem}
\usepackage{multicol}

\usepackage{enumerate}
\usepackage{breqn}
\usepackage{parskip}

\usepackage{amsmath, amsthm, amsfonts, amssymb}
\usepackage{graphicx}
\usepackage{tikz}
\usetikzlibrary{arrows,calc,patterns}
\usepackage{pgfplots}
\pgfplotsset{compat=newest}
\usepackage{url}
\usepackage{multicol}
\usepackage{thmtools}

\usepackage{caption}
\usepackage{subcaption}

\usepackage{pifont}

% MATH commands
\newcommand{\bC}{\mathbb{C}}
\newcommand{\bR}{\mathbb{R}}
\newcommand{\bN}{\mathbb{N}}
\newcommand{\bZ}{\mathbb{Z}}
\newcommand{\bT}{\mathbb{T}}
\newcommand{\bD}{\mathbb{D}}

\DeclareMathOperator{\dom}{dom}

\newcommand{\spc}{\vspace*{0.5cm}}
\CorrectChoiceEmphasis{\color{red}}

\begin{document}
\noindent \hrulefill \\
	MATH-241 Calculus I \hfill Created by Rukiyah Walker\\
	Homework 5 \hfill Spring 2023\\ \vspace*{-1cm}
 
	\noindent\hrulefill

\qformat{\rule{0.3\textwidth}{.4pt} \begin{large}{\textsc{Question}} \thequestion \end{large} \hspace*{0.2cm} \hrulefill \hspace*{0.1cm} \textbf{(\totalpoints\hspace*{0.1cm} pts)}}

\begin{questions}

\vspace*{1cm}

\question[1]

Suppose you have a function, $f(x)$, that represents some curve, and a point $(a, b)$ on the curve. How do you find the slope of the tangent line at that point?

\begin{choices}
\choice Plug in the given point, and write in $y = mx + b$ form.
\choice Find the derivative of $f(x)$.
\choice Re-write the function in point-slope form.
\CorrectChoice Find the derivative of $f(x)$, and then determine the value of the derivative at that point.
\end{choices}

\spc

\question[1]

If $c$ is a constant and $f$ is a differentiable function, then $\frac{d}{dx}[cf(x)]$ is equivalent to:

\begin{choices}
\CorrectChoice $c\frac{d}{dx}f(x)$
\choice $\frac{d}{dx}cf(x)$
\choice $\frac{d}{dx}f(x)c$
\choice $f(x)$
\end{choices}

\spc

\question[1]

If $f$ and $g$ are both differentiable functions, then $\frac{d}{dx}[f(x)g(x)]$ is equivalent to:

\begin{choices}
\choice $\frac{d}{dx}f(x)\frac{d}{dx}g(x)$
\choice $\frac{d}{dx}[f(x) + g(x)]$
\CorrectChoice $f(x)[\frac{d}{dx}g(x)] + g(x)[\frac{d}{dx}f(x)]$
\choice $\frac{d}{dx}f(x) + \frac{d}{dx}g(x)$
\end{choices}

\spc

\question[1]

If $f$ and $g$ are both differentiable functions, then $\frac{d}{dx}[\frac{f(x)}{g(x)}]$ is equivalent to:

\begin{choices}
\choice $\frac{\frac{d}{dx}f(x)}{\frac{d}{dx}g(x)}$
\CorrectChoice $\frac{g(x)\frac{d}{dx}[f(x)] - f(x)\frac{d}{dx}[g(x)]}{[g(x)^2]}$
\choice $\frac{\frac{d}{dx}f(x) - \frac{d}{dx}g(x)}{f(x)g(x)}$
\choice $\frac{d}{dx}f(x) - \frac{d}{dx}g(x)$
\end{choices}

\spc

\question[1]

Two lines are parallel if:

\begin{choices}
\CorrectChoice They have the same slope.
\choice They are continuous.
\choice They have the same y-intercept.
\choice They pass the vertical line test. 
\end{choices}

\spc

\question[1]

Match the functions below (A-D) with their derivatives (E-H). 

(Write your answers as A-G, B-F, etc.)

\begin{multicols}{2}
\begin{choices}
\choice $\frac{d}{dx}csc(x)$
\choice $\frac{d}{dx}sec(x)$
\choice $\frac{d}{dx}cot(x)$
\choice $\frac{d}{dx}cos(x)$
\choice $-csc(x)cot(x)$
\choice $-csc^2(x)$
\choice $-sin(x)$
\choice $sec(x)tan(x)$
\end{choices}
\end{multicols}

\spc

\question[1]

$f(x) = cos(x)$. What is $f'(\frac{\pi}{2})$?

\begin{choices}
\choice 1
\choice 0 
\CorrectChoice -1
\choice $-sin(x)$

\end{choices}

\spc

\question[1]

$f(x) = sin(x)$. What is $f'(\frac{\pi}{4})$?

\begin{choices}
\choice 0
\CorrectChoice $\frac{\sqrt{2}}{2}$
\choice 1
\choice $-\frac{\sqrt{2}}{2}$
\end{choices}

\newpage

\question[1]

Given a function $f(x)$, when would we have a horizontal tangent line?

\begin{choices}
\CorrectChoice When $f'(x) = 0$.
\choice When the function is continuous. 
\choice When $x = 0$.
\choice When $f(x) = 0$.

\end{choices}

\question[1]

Calculate $\lim_{x \to 0} \frac{sin^2(x)}{x}$

\begin{choices}
\choice 1
\choice $sin(x)$
\choice -1
\CorrectChoice 0
\end{choices}


\end{questions}

\end{document}