\include{../../semesterTerm.tex}

\begin{document}
	\noindent \hrulefill \\
	MATH-241 \hfill Pierre-Olivier Paris{\'e}\\
	Solutions Section 5-3 \hfill \semester \\\vspace*{-1cm}
	
	\noindent\hrulefill
	
	\spc
	
	\exo{16}
	\\
	Here is the sketch of the region to rotate about the axis $x = 3$:
		\begin{figure}[h]
		\begin{subfigure}[b]{0.45\textwidth}
		\centering
		\includegraphics[scale=0.3]{5-3_exo16}
		\caption{Region to rotate}
		\end{subfigure}
		\begin{subfigure}[b]{0.45\textwidth}
		\animategraphics[scale=0.75]{20}{python_5-3_16/Revolution}{0}{200}
		\caption{Rotation of the region}
		\end{subfigure}
		\end{figure}
	
	We will use the cylindrical shells method. The radius is $x + 1$ and the height is $y$ and the limits are $0 \leq x \leq 2$. Thus, we get
		\begin{align*}
		V(S) = \int_0^2 2\pi (x + 1) y \, dx = 2\pi \int_0^2 (x + 1) (4 - 2x) \, dx = 20/3 .
		\end{align*}
	
	\newpage
	
	\exo{18}
	\\
	The region is bounded by the curves
		\begin{align*}
		y = \sqrt{x} \quad \text{ and } \quad x = 2y .
		\end{align*}
	Therefore, the curve meets when
		\begin{align*}
		\sqrt{x} = \frac{x}{2} \iff x = \frac{x^2}{4} \iff \frac{1}{4} x (x- 4) = 0 \iff x = 0 \text{ or } x = 4 .
		\end{align*}
	A sketch of the region is presented below with a typical rectangle to generate the spherical shell:
	\begin{center}
	\includegraphics[scale=0.4]{fig5.png}
	\end{center}
	
	After rotating about the line $x = 5$, we obtain a cylindrical shell with
		\begin{itemize}
		\item height: $\sqrt{x} - \frac{x}{2}$;
		\item radius: $5 - x$;
		\item thickness: $dx$.
		\end{itemize}
	Therefore, the volume is given by
		\begin{align*}
		\int_a^b 2\pi (\text{radius}) (\text{height}) \, dx = \int_0^4 2\pi (5-x) \big(\sqrt{x} - \frac{x}{2} \big) \, dx
		\end{align*}
	The value of this integral is the volume of the solid of revolution. Therefore, the volume of the solid of revolution is $\frac{136}{15} \pi$.
	
	\newpage
	
	\exo{30}
	\\
	The radius is $y$ and the height is $x = \sqrt{y - 1}$ and the limits are $1 \leq y \leq 5$. So, the solid is obtained by rotating the following region around the $x$ axis:
		\begin{figure}[h]
		\centering
		\includegraphics[scale=0.3]{5-3_exo30}
		\end{figure}
	
\end{document}