\documentclass[addpoints, 12pt]{exam}%, answers]
\usepackage[utf8]{inputenc}
\usepackage[T1]{fontenc}

\usepackage{lmodern}
\usepackage{arydshln}
\usepackage[margin=2cm]{geometry}

\usepackage{enumitem}
\usepackage{multicol}

\usepackage{enumerate}
\usepackage{breqn}
\usepackage{parskip}

\usepackage{amsmath, amsthm, amsfonts, amssymb}
\usepackage{graphicx}
\usepackage{tikz}
\usetikzlibrary{arrows,calc,patterns}
\usepackage{pgfplots}
\pgfplotsset{compat=newest}
\usepackage{url}
\usepackage{multicol}
\usepackage{thmtools}

\usepackage{caption}
\usepackage{subcaption}

\usepackage{pifont}

% MATH commands
\newcommand{\bC}{\mathbb{C}}
\newcommand{\bR}{\mathbb{R}}
\newcommand{\bN}{\mathbb{N}}
\newcommand{\bZ}{\mathbb{Z}}
\newcommand{\bT}{\mathbb{T}}
\newcommand{\bD}{\mathbb{D}}

\DeclareMathOperator{\dom}{dom}

\newcommand{\spc}{\vspace*{0.5cm}}
\CorrectChoiceEmphasis{\color{red}}

\begin{document}
\noindent \hrulefill \\
	MATH-241 Calculus I \hfill Created by Rukiyah Walker\\
	Homework 9 \hfill Spring 2023\\ \vspace*{-1cm}
 
	\noindent\hrulefill

\qformat{\rule{0.3\textwidth}{.4pt} \begin{large}{\textsc{Question}} \thequestion \end{large} \hspace*{0.2cm} \hrulefill \hspace*{0.1cm} \textbf{(\totalpoints\hspace*{0.1cm} pts)}}

\begin{questions}

\vspace*{0.5cm}

\question[1]

Fill in the blank: 

If the graph of $f$ lies above all of its tangents on an interval $I$, it is called ____. 
\newline If the graph of $f$ lies below all of its tangents on an interval $I$, it is called ____.

\begin{multicols}{2}
\begin{choices}
\choice An inflection point.
\newline A non-inflection point.
\choice Concave downward.
\newline Concave upward.
\CorrectChoice Concave upward.
\newline Concave downward.
\choice $f'$ exists.
\newline $f''$ exists.
\end{choices}
\end{multicols}

\spc

\question[1]

Suppose we have a point $P$ on a curve $f(x)$. If $P$ is an inflection point, what does that mean for $f$?

\begin{choices}
\choice $f''$ is continuous.
\CorrectChoice $f$ is continuous at $P$ and the curve changes from concave upward to concave downward (or concave downward to concave upward). 
\choice $f'$ is continuous.
\choice The graph of $f$ is concave upward.
\end{choices}

\spc

\question[1]

Suppose we have a function $f$. If the graph of $f$ is concave downward on some interval $I$, what does that tell us about $f''(x)$?

\begin{multicols}{2}
\begin{choices}
\CorrectChoice $f''(x) < 0$ for all $x$ in $I$.
\choice $f'(x) > 0$ for all $x$ in $I$.
\choice $f''(x) > 0$ for all $x$ in $I$.
\choice $f'(x) < 0$ for all $x$ in $I$.
\end{choices}
\end{multicols}

\spc

\question[1]

Suppose $f''$ is continuous near $c$.
Fill in the blank:

If $f'(c) = 0$ and $f''(c) > 0$, then ____. If $f'(c) = 0$ and $f''(c) < 0$, then ____.

\begin{multicols}{2}
\begin{choices}
\choice $f$ has an absolute minimum at $c$.
\newline $f$ has an absolute maximum at $c$.
\choice $f$ is concave upward.
\newline $f$ is concave downward.
\choice $f$ has a local maximum at $c$.
\newline $f$ has a local minimum at $c$.
\CorrectChoice $f$ has a local minimum at $c$.
\newline $f$ has a local maximum at $c$.
\end{choices}
\end{multicols}

\spc

\question[1]

The concavity test only has cases where $f''(c) < 0$ and $f''(c) > 0$ for the graph $f$. What do you generally expect to happen if $f'(c) = 0$?

\begin{choices}
\choice Concave upward.
\CorrectChoice There is an inflection point.
\choice The second derivative does not exist.
\choice Concave downward.
\end{choices}

\spc

\question[1]

$\lim_{x \to \infty} \frac{3x^2 -2}{x^2 + 1} = 3$. This means that the line $y = 3$ is a/an ____.

\begin{choices}
\choice Inflection point.
\choice Local maximum.
\choice Local minimum.
\CorrectChoice Horizontal asymptote. 
\end{choices}

\spc

\question[1]

When will the function $f(x) = \frac{x^2 - 1}{2x-3}$ have a vertical asymptote?

\begin{multicols}{2}
\begin{choices}
\CorrectChoice at $x = 3/2$
\choice at $x = 1$
\choice at $x = \infty$
\choice There is no vertical asymptote.
\end{choices}
\end{multicols}

\spc

\question[1]

What do the following statements all have in common?

$\lim_{x \to -\infty} f(x) = \infty$, $\lim_{x \to -\infty} f(x) = -\infty$, $\lim_{x \to \infty} f(x) = -\infty$

\begin{choices}
\choice The limits do not exist.
\CorrectChoice The values of $f(x)$ become arbitrarily large (positive or negative) as we let $x$ become arbitrarily large (positive or negative).
\choice The limit is not defined since we will have complex numbers in the answer.
\choice There is a horizontal asymptote at $y = 0$.
\end{choices}

\newpage

\question[1]

Let $f(x) = \frac{1}{x^r}$. When is $\lim_{x \to \infty} f(x)$ not defined?

\begin{multicols}{2}
\begin{choices}
\choice When $x \leq 0$.
\choice When $x$ goes to $\infty$
\choice When $x$ goes to $-\infty$.
\CorrectChoice When $ r \leq 0$.
\end{choices}
\end{multicols}

\question[1]

Let $f(x) = 1 + \frac{1}{x} + \frac{1}{x^2} - \frac{2}{x^3} + \frac{3}{x^5}$. Evaluate $\lim_{x \to \infty} f(x)$.

\begin{multicols}{2}
\begin{choices}
\choice $\infty$
\choice 8
\CorrectChoice 1
\choice $\nexists$
\end{choices}
\end{multicols}

\spc

\end{questions}
\end{document}