\documentclass[12pt]{amsart}

\usepackage{xcolor}
\usepackage{hyperref}
\usepackage[margin=1.1in, tmargin=1in, bmargin=1in]{geometry}
\usepackage{footmisc}
\usepackage{multicol}

\pagestyle{myheadings}
%\markleft{Syllabus for Math 331, Spring 2020}
%\markright{Syllabus for Math 331, Spring 2020}

\newcommand{\spacer}{\vspace{.2cm}}
\newcommand{\svs}{\vspace{.1cm}}

\newcommand{\red}[1]{\textcolor{red}{#1}}
\definecolor{gold}{rgb}{0.80,0.68,0.00}\newcommand{\gold}[1]{\textcolor{gold}{#1}}

\begin{document}
\thispagestyle{empty}

\begin{center}
\textsc{Math 241} \hfill {\Large\textsc{Syllabus}} \hfill \textsc{Spring 2023}
\end{center}

\noindent\hrulefill

\noindent\textbf{Instructions Info. Sections 06--07} \hfill \textbf{Room} \\
\noindent Lecture: TR 9:00--10:15am \hfill Keller 302\\
Recitation Section 6: F 9:30--10:20am \hfill Keller 401\\
Recitation Section 7: F 10:30--11:20am \hfill Keller 401

\noindent\hrulefill

\spacer

\noindent\textbf{Instructor:} Pierre-Olivier Paris{\'e} (email: \texttt{parisepo@hawaii.edu})\\
Office: Physical Science Building (PSB) 302\\
%Zoom Office: 976 6163 4762 (ID), 359469 (Passcode)\\
Office hours: TR 2:30pm--4:00pm (starting January 17th)

\spacer

\noindent{\bf Teaching Assistant (TA)}: Rukiyah Walker (email: \texttt{rukiyahw@hawaii.edu})\\
Office: Keller 301D\\
Office hours: MW 12:30pm--2:00pm

\noindent\hrulefill

\section*{Course description}
Review of precalculus material such as algebra, analytic geometry, functions, plotting graphs of functions, basic operation with functions, difference quotients. Limits and continuity. Derivative and applications. Integration and basic applications.\svs

\noindent{\bf Prerequisites:}
A grade of C or better in Math 140 or Math 215 or precalculus assessment as specified by the department.

\section*{Course material}

\noindent{\bf Book:} \emph{Calculus}, 8th Edition, by Stewart (ISBN-13: 9781285740621). 

\noindent{\bf Course website:} \url{https://mathopo.ca/courses-website/math-241/math-241}. All the information about the course (like the schedule, lecture notes, homework solutions, and important dates) is posted on the course website.

\noindent{\bf Online ressources:} A MultiTerm e-Pack (WebAssign and online text for up to 4 years) is
available directly from the publisher, Cengage. You can check at the UH bookstore for a lower price.

\noindent{\bf Note:} This course space is being shared by students in the following CRNs [82605, 82612] and if you do not wish to be in a shared course space, please drop this section and register for another section or alternate course.

\section*{Important Dates}
%Please make sure you write down somewhere these important dates:
Refer to the google calendar on the course website for a detailed \href{https://mathopo.ca/courses-website/math-241/schedule}{schedule}.
	\begin{itemize}
	\item Exams:
		\begin{itemize}
		\item[--] Midterm 1: February, 22th, 6:00pm--7:15pm.
		\item[--] Midterm 2: April, 12th, 6:00pm--7:15pm.
		\item[--] Final: May, 10th, 12:00pm--2:00pm.
		\end{itemize}
	\item Non-instructional day(s):
		\begin{itemize}
		\item[--] Spring break: March, 13th--17th.
		\end{itemize}
	\end{itemize}
	
\newpage

\section*{Grading components}
Your course average will be determined by a weighted average of the components below.

\begin{enumerate}
\item {\bf Midterm exams (40\%):} There will be two(2) common midterm exams. Midterms are not comprehensive and are in-person. 
\item {\bf Final exam (30\%):} There will be a common final exam as scheduled by the university. This is a comprehensive exam and it will be in-person.
\item{\bf Homework (10\%):} There will be homework assigned each Fridays (starting on January, 20th) and due for the next Tuesdays, before 9:00am. The homework problems will be posted on Laulima, in Assignments. You will be required to enter your answers in a textfield as illustrated in the \href{https://mathopo.ca/courses-website/math-241/Homework.html}{homework webpage}. The solutions will be posted on the course website, on the \href{https://mathopo.ca/courses-website/math-241/Homework.html}{homework webpage}. The best 12 scores will be used to compute your average score for this component.
\item{\bf Recitations Attendance (20\%):} At the end of each recitation, we will take attendance. If you were there, you will get 1/1 for the recitation and if you were not there, you will get 0/1. The best 14 scores will be used for your average. Recitations are an opportunity to work on the exercises in each sections in the \href{https://mathopo.ca/courses-website/math-241/Lecture-Notes.html}{Lecture Notes webpage}.
\end{enumerate}

\noindent{\bf Notes:} 
	\begin{itemize}
	\item The first homework will be available on January, 10th and will be due before January 16th, 11:55pm.
	\item If you follow ALL the rules enumerated in the homework page, you will earn a bonus of 5\%.
	\end{itemize}

%\begin{table}[h]
%\begin{tabular}{c|c|c}
%Evaluation & Number & \% average \\ \hline\hline
%Midterms & 2 & 40\% \\\hline
%Final & 1 & 30\% \\\hline
%Homework & 12 & 15\% \\\hline
%Worksheets & 12 & 15\% \\\hline\hline
%Total & 27 & 100\%
%\end{tabular}
%\end{table}

\section*{Lectures and recitals}
All students are expected to attend class in-person, participate in their respective recitation and take their own personal notes. The annotated lecture notes will be posted on the Laulima, in the webcontent. 

If you miss a lecture or recital, you are responsible for any assignments and/or announcements made. Unavoidable absences should be explained to the instructor. Office hours will not be utilized to re-teach material. It is your responsibility to stay up-to-date.  

\section*{Missed assignment policies}
\noindent{\bf Policies for exams:}  Attendance on the exams is compulsory; otherwise, a grade of zero will be recorded. Any student who has an excused, documented conflict with a test time must inform their instructor \textbf{within the two weeks prior to the midterm targeted} when possible.  Late requests will either be denied or will result in an automatic deduction from the exam score. An absence due to a positive COVID-19 test result must be notified to the instructor with a doctor note and is not subject to the previous 2 weeks rule.

For those students with an excused absence for a midterm, there will be a make-up exam which must be taken within two working days of the scheduled exam time. \textbf{The final exam cannot be taken before the scheduled time for any reason.}

Conflicts arising from work or social obligations, or from personal travel plans do \textbf{not} qualify as excused absences. By registering for this course, you are agreeing to take all exams at the scheduled times.

Past final exams: \url{http://math.hawaii.edu/home/common-finals/241/}
\svs

\noindent{\bf Policies for homework:} No late homework will be accepted. The mark zero(0) will be attributed to a late homework. For each homework, there will be 5\% of bonus points for respecting the template.\svs

\noindent{\bf Academic integrity:}
All students are expected to abide by the university's Conduct Code. Academic sanctions for dishonesty may include receiving an F in the assignment or receiving an F in the class. There may be additional administrative sanctions, see
\newline {\url{https://www.hawaii.edu/policy/index.php?action=home&policySection=ep}}
\svs

\noindent{\bf Academic Expectations:} You should familiarize yourself with the  academic expectations for UH Math courses outlined here: 
\newline \url{http://www.math.hawaii.edu/~dale/Expectations.html}.

\section*{Classroom policies}
Tablets can be used for note-taking only. During lectures, refrain from using electronic items, cell phones, music players, tablets, and laptops for any other purposes than note-taking. The use of electronic device other than for note-taking may disrupt the class.

Please arrive, be seated and ready to start each class on time. If you have a valid reason to leave early, please advise me before the class and try to sit near the exit to minimize disruption.

\section*{Sources of help}
{\bf KOKUA:} I am happy to work with you and the KOKUA Program (Office for Students with Disabilities), if you need course accommodations due to a disability. KOKUA can be reached at (808) 956-7511 or (808) 956-7612 (voice/text) in room 013 of the Queen Lili`uokalani Center for Student Services. All course modifications must be arranged through KOKUA. You are encouraged to start this process as early as possible.\svs

{\bf More:} There is also tutoring available at the \textit{Learning Emporium}, hosted by the College of Natural Sciences. Details will be posted when available.

\section*{Concerns}
If at any time during the semester you have any questions or concerns about the class, please contact me during regularly scheduled office hours or via email to make an appointment. If you feel it's a bigger issue and you would like to discuss it with other faculty of the UH Math department, you may also contact the following people:
\spacer

\noindent {\bf Director of Undergraduate Studies}\\
Mirjana Jovovic \\
Email: \texttt{undergrad-dir@math.hawaii.edu}

\svs
\noindent {\bf Associate Chair}\\
Bj{\o}rn Kjos-Hanssen \\
Email: \texttt{assoc-chair@math.hawaii.edu}

\end{document}

