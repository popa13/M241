\include{../../semesterTerm.tex}

\begin{document}
	\noindent \hrulefill \\
	MATH-241 \hfill Pierre-Olivier Paris{\'e}\\
	Solutions Section 1-2 \hfill \semester \\\vspace*{-1cm}
	
	\noindent\hrulefill
	
	\spc
	
	\exo{6}
	\\
	The denominator must never vanish. So we find where $1 - \tan x = 0$. This occurs when $1 = \tan x$, which is equivalent to $x = \pi/4 + k \pi$ where $k$ is any integer. Also, the domain of the $\tan$ function is $(- \infty , \infty ) \backslash \{ \pi/2 + k \pi \, : \, k \in \mathbb{Z} \}$. So $\Dom (f) = (-\infty , \infty ) \backslash \{ \pi/4 + k \pi , \pi/2 + k \pi \, : \, k \in \mathbb{Z} \}$.
	
	
	
\end{document}