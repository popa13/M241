\documentclass[addpoints, 12pt]{exam}%, answers]
\usepackage[utf8]{inputenc}
\usepackage[T1]{fontenc}

\usepackage{lmodern}
\usepackage{arydshln}
\usepackage[margin=2cm]{geometry}

\usepackage{enumitem}
\usepackage{multicol}

\usepackage{enumerate}
\usepackage{breqn}
\usepackage{parskip}

\usepackage{amsmath, amsthm, amsfonts, amssymb}
\usepackage{graphicx}
\usepackage{tikz}
\usetikzlibrary{arrows,calc,patterns}
\usepackage{pgfplots}
\pgfplotsset{compat=newest}
\usepackage{url}
\usepackage{multicol}
\usepackage{thmtools}

\usepackage{caption}
\usepackage{subcaption}

\usepackage{pifont}

% MATH commands
\newcommand{\bC}{\mathbb{C}}
\newcommand{\bR}{\mathbb{R}}
\newcommand{\bN}{\mathbb{N}}
\newcommand{\bZ}{\mathbb{Z}}
\newcommand{\bT}{\mathbb{T}}
\newcommand{\bD}{\mathbb{D}}

\DeclareMathOperator{\dom}{dom}

\newcommand{\spc}{\vspace*{0.5cm}}
\CorrectChoiceEmphasis{\color{red}}

\begin{document}
\noindent \hrulefill \\
	MATH-241 Calculus I \hfill Created by Rukiyah Walker\\
	Homework 13 \hfill Spring 2023\\ \vspace*{-1cm}
 
	\noindent\hrulefill

\qformat{\rule{0.3\textwidth}{.4pt} \begin{large}{\textsc{Question}} \thequestion \end{large} \hspace*{0.2cm} \hrulefill \hspace*{0.1cm} \textbf{(\totalpoints\hspace*{0.1cm} pts)}}

\begin{questions}

\vspace*{0.5cm}

\question[1]

Express the limit, $lim_{n \to \infty} \sum_{i=1}^{n}  \frac{2}{n} \big( 1 + \frac{2i}{n} \big) \sqrt{1 + \frac{2i}{n}}$, as a definite integral. 

\begin{multicols}{2}
\begin{choices}
\CorrectChoice $\int_{1}^{3} x\sqrt{x} \,dx$
\choice $\int_{1}^{3} \sqrt{x} \,dx$
\choice $\int_{1}^{3} (x\sqrt{x})\frac{2}{n} \,dx$
\choice $\int_{1}^{n} x\sqrt{x} \,dx$
\end{choices}
\end{multicols}

\spc

\question[1]

Suppose $\int_{1}^{2} x^2 \,dx$. Using this information, what are absolute minimum and maximum of the integrand $f(x) = x^2$? (Write as an inequality).

\begin{multicols}{2}
\begin{choices}
\choice $1 \leq x \leq 2$
\CorrectChoice $1 \leq x^2 \leq 4$
\choice $1 \leq x^2 \leq 2$
\choice $1 \leq x \leq 4$
\end{choices}
\end{multicols}

\spc

\question[1]

Suppose you want to estimate $\int_{1}^{4} f(x) \,dx$, with 3 rectangles. What are the midpoints of the subintervals?

\begin{multicols}{2}
\begin{choices}
\choice $x_1 = 2, x_2 = 3, x_3 = 4$
\choice $x_1 = \frac{1}{2}, x_2 = \frac{3}{2}, x_3 = \frac{5}{2}$
\CorrectChoice $x_1 = \frac{3}{2}, x_2 = \frac{5}{2}, x_3 = \frac{7}{2}$
\choice $x_1 = 1, x_2 = 2, x_3 = 3$
\end{choices}
\end{multicols}


\question[1]

Which of the following defines a definite integral?

\begin{multicols}{2}
\begin{choices}
\choice An antiderivative that produces a function with an arbitrary constant $C$.
\choice The limit as $f(x)$ goes to infinity.
\choice The derivative of the area function.
\CorrectChoice An integral which is evaluated over a specific interval, and produces a constant value.
\end{choices}
\end{multicols}


\question[1]
The fundamental theorem of calculus (part 1) says: If $f$ is continuous on $[a, b]$, then the function $g$ defined by $g(x) = \int_{a}^{x} f(t) \,dt$, $a \leq x \leq b$, is continuous on $[a, b]$ and differentiable on $(a, b)$, and $g'(x) = f(x).$
What does this mean in words?

\begin{multicols}{2}
\begin{choices}
\choice $g'(x)$ exists on $[a, x]$.
\CorrectChoice The derivative of the area function is equal to the integrand.
\choice The derivative of $f(x)$ is equal to $F(b) - F(a)$.
\choice $g(x)$ is only differentiable when $f(t)$ exists on $[a, b]$.
\end{choices}
\end{multicols}

\newpage

\question[1]

The fundamental theorem of calculus (part 2) says: If $f(x)$ is continuous on $[a, b]$, then $\int_{a}^{b} f(x) \,dx = F(b) - F(a)$, where $F(x)$ is any antiderivative of $f(x)$.
What does this mean in words?

\begin{choices}
\CorrectChoice If $F(x)$ is the anti-derivative of $f(x)$, then $\int_{a}^{b} f(x) \,dx$ is equal to $F(x)$ evaluated at $b$, subtracted by $F(x)$ evaluated at $a$.
\choice The derivative of $f(x)$ is equal to $F(b) - F(a)$.
\choice The derivative of the area function is equal to the integrand.
\choice The derivative of $f(x)$ is equal to $F(x)$ evaluated at $b$, subtracted by the $F(x)$ evaluated at $a$.
\end{choices}

\spc

\question[1]

Suppose you have $\int_{-1}^{2} x^2 \,dx$ and $F(x) + C$ is the antiderivative of $f(x)$, where $C = 0$.
Using the same notation as the Fundamental Theorem of Calculus, what is $F(b)$ and $F(a)$?

\begin{multicols}{2}
\begin{choices}
\choice $F(b) = F(-1) = \frac{-1}{3}$
\newline
$F(a) = F(2) = \frac{8}{3}$
\choice $F(b) = F(2) = \frac{4}{3}$
\newline 
$F(a) = F(-1) = \frac{-1}{3}$
\CorrectChoice $F(b) = F(2) = \frac{8}{3}$ 
\newline
$F(a) = F(-1) = \frac{-1}{3}$
\choice $F(b) = F(2) = \frac{8}{3}$
\newline
$F(a) = F(-1) = \frac{1}{3}$
\end{choices}
\end{multicols}

\spc

\question[1]

Which of the below is equivalent to $\int_{1}^{2} x^3 + 2x + 3 \,dx$?

\begin{multicols}{2}
\begin{choices}
\choice $\frac{39}{4}$
\choice $\frac{x^4}{4} + x^2 + 3x \Big|_1^2$
\choice $(\frac{2^4}{4} + 2^2 + 3(2)) - (\frac{1}{4} + 1 + 3)$
\CorrectChoice A, B, and C.
\end{choices}
\end{multicols}

\spc

\question[1]

Evaluate $\int_{2}^{2} \sqrt{x} \,dx$.

\begin{multicols}{2}
\begin{choices}
\CorrectChoice 0
\choice $\frac{2^{5/2}}{3}$
\choice Does not exist.
\choice $\frac{2^{5/2}-2}{3}$
\end{choices}
\end{multicols}

\spc

\question[1]

Evaluate $\int_{-1}^{2} \frac{1}{x^3} \,dx$.

\begin{multicols}{2}
\begin{choices}
\choice $\frac{3}{8}$
\CorrectChoice Does not exist
\choice 0
\choice $\frac{-3}{8}$
\end{choices}
\end{multicols}

\end{questions}
\end{document}

the derivative of the area function is equal to the integrand