\documentclass[addpoints, 12pt]{exam}%, answers]
\usepackage[utf8]{inputenc}
\usepackage[T1]{fontenc}

\usepackage{lmodern}
\usepackage{arydshln}
\usepackage[margin=2cm]{geometry}

\usepackage{enumitem}

\usepackage{amsmath, amsthm, amsfonts, amssymb}
\usepackage{graphicx}
\usepackage{tikz}
\usetikzlibrary{arrows,calc,patterns}
\usepackage{pgfplots}
\pgfplotsset{compat=newest}
\usepackage{url}
\usepackage{multicol}
\usepackage{thmtools}
\usepackage{wrapfig}

\usepackage{caption}
\usepackage{subcaption}

\usepackage{pifont}

% MATH commands
\newcommand{\bC}{\mathbb{C}}
\newcommand{\bR}{\mathbb{R}}
\newcommand{\bN}{\mathbb{N}}
\newcommand{\bZ}{\mathbb{Z}}
\newcommand{\bT}{\mathbb{T}}
\newcommand{\bD}{\mathbb{D}}

\newcommand{\cL}{\mathcal{L}}
\newcommand{\cM}{\mathcal{M}}
\newcommand{\cP}{\mathcal{P}}
\newcommand{\cH}{\mathcal{H}}
\newcommand{\cB}{\mathcal{B}}
\newcommand{\cK}{\mathcal{K}}
\newcommand{\cJ}{\mathcal{J}}
\newcommand{\cU}{\mathcal{U}}
\newcommand{\cO}{\mathcal{O}}
\newcommand{\cA}{\mathcal{A}}
\newcommand{\cC}{\mathcal{C}}
\newcommand{\cF}{\mathcal{F}}

\newcommand{\fK}{\mathfrak{K}}
\newcommand{\fM}{\mathfrak{M}}

\newcommand{\ga}{\left\langle}
\newcommand{\da}{\right\rangle}
\newcommand{\oa}{\left\lbrace}
\newcommand{\fa}{\right\rbrace}
\newcommand{\oc}{\left[}
\newcommand{\fc}{\right]}
\newcommand{\op}{\left(}
\newcommand{\fp}{\right)}

\newcommand{\ra}{\rightarrow}
\newcommand{\Ra}{\Rightarrow}

\renewcommand{\Re}{\mathrm{Re}\,}
\renewcommand{\Im}{\mathrm{Im}\,}
\newcommand{\Arg}{\mathrm{Arg}\,}
\newcommand{\Arctan}{\mathrm{Arctan}\,}
\newcommand{\sech}{\mathrm{sech}\,}
\newcommand{\csch}{\mathrm{csch}\,}
\newcommand{\Log}{\mathrm{Log}\,}
\newcommand{\cis}{\mathrm{cis}\,}

\newcommand{\ran}{\mathrm{ran}\,}
\newcommand{\bi}{\mathbf{i}}
\newcommand{\Sp}{\mathrm{span}\,}
\newcommand{\Inv}{\mathrm{Inv}\,}
\newcommand\smallO{
  \mathchoice
    {{\scriptstyle\mathcal{O}}}% \displaystyle
    {{\scriptstyle\mathcal{O}}}% \textstyle
    {{\scriptscriptstyle\mathcal{O}}}% \scriptstyle
    {\scalebox{.7}{$\scriptscriptstyle\mathcal{O}$}}%\scriptscriptstyle
  }
\newcommand{\HOL}{\mathrm{Hol}}
\newcommand{\cl}{\mathrm{clos}}
\newcommand{\ve}{\varepsilon}

\DeclareMathOperator{\dom}{dom}

%%%%%% Définitions Theorems and al.
%\declaretheoremstyle[preheadhook = {\vskip0.2cm}, mdframed = {linewidth = 2pt, backgroundcolor = yellow}]{myThmstyle}
%\declaretheoremstyle[preheadhook = {\vskip0.2cm}, postfoothook = {\vskip0.2cm}, mdframed = {linewidth = 1.5pt, backgroundcolor=green}]{myDefstyle}
%\declaretheoremstyle[bodyfont = \normalfont , spaceabove = 0.1cm , spacebelow = 0.25cm, qed = $\blacktriangle$]{myRemstyle}

%\declaretheorem[ style = myThmstyle, name=Th\'eor\`eme]{theorem}
%\declaretheorem[style =myThmstyle, name=Proposition]{proposition}
%\declaretheorem[style = myThmstyle, name = Corollaire]{corollary}
%\declaretheorem[style = myThmstyle, name = Lemme]{lemma}
%\declaretheorem[style = myThmstyle, name = Conjecture]{conjecture}

%\declaretheorem[style = myDefstyle, name = D\'efinition]{definition}

%\declaretheorem[style = myRemstyle, name = Remarque]{remark}
%\declaretheorem[style = myRemstyle, name = Remarques]{remarks}

\newtheorem{theorem}{Théorème}
\newtheorem{corollary}{Corollaire}
\newtheorem{lemma}{Lemme}
\newtheorem{proposition}{Proposition}
\newtheorem{conjecture}{Conjecture}

\theoremstyle{definition}

\newtheorem{definition}{Définition}[section]
\newtheorem{example}{Exemple}[section]
\newtheorem{remark}{\textcolor{red}{Remarque}}[section]
\newtheorem{exer}{\textbf{Exercice}}[section]


\tikzstyle{myboxT} = [draw=black, fill=black!0,line width = 1pt,
    rectangle, rounded corners = 0pt, inner sep=8pt, inner ysep=8pt]

\begin{document}
	\noindent \hrulefill \\
	\noindent MATH-241 \hfill Created by Pierre-O. Paris{\'e}\\
	Midterm 02 \hfill 2023/12/04, Spring 2023\\\vspace*{-0.7cm}

\noindent\hrulefill
	
\vspace*{1cm}

\noindent\makebox[\textwidth]{\textbf{Last name:}\enspace \hrulefill}

\vspace*{0.5cm}

\noindent\makebox[\textwidth]{\textbf{First name:}\enspace\hrulefill}

\vspace*{0.5cm}

\noindent\makebox[\textwidth]{\textbf{Section:}\enspace\hrulefill}

\vspace*{1cm}

\noindent\textbf{Instructions:} 

\begin{itemize}
\item Make sure to write your complete name on your copy. 
\item You must answer all eight (8) questions below and write your answers directly on the questionnaire.
\item You have 75 minutes to complete the exam.
\item When you are done (or at the end of the 75min period), return your copy. 
\item Devices such as smartphones, cellphones, laptops, tablets, e-readers, ipods, gameboys (and, you know, any other electronic devices that I haven't thought of) may not be used during the exam. 
\item You can not use a calculator.
\item \textbf{Turn off your cellphones during the exam}.
\item Lecture notes and the textbook are not allowed during the exam. 
\item You must show ALL your work to have full credit. An answer without justification is worth no points (except if it is mentioned explicitly in the question not to justify).
\item Draw a square around your final answer.
\end{itemize}

\vspace{1cm}

\noindent\textbf{Your Signature:} \hrulefill

\vspace*{1.5cm}
\noindent \textsc{May the Force be with you!} \hfill \textsc{Pierre Parisé}

\vspace*{0.5cm}

\begin{center}
\begin{minipage}{0.29\textwidth}
\begin{Huge}
\textsc{University of Hawai'i}
\end{Huge}
\end{minipage}
\begin{minipage}{0.12\textwidth}
\includegraphics[scale=0.05]{../../../../manoaseal_transparent.png}
\end{minipage}
\end{center}

\qformat{\rule{0.3\textwidth}{.4pt} \begin{large}{\textsc{Question}} \thequestion \end{large} \hspace*{0.2cm} \hrulefill \hspace*{0.1cm} \textbf{(\totalpoints\hspace*{0.1cm} pts)}}

\vspace*{0.5cm}

\newpage % End of cover page

\begin{questions}

\question[10]
The volume of a cube is increasing at a rate of $10 \text{cm}^3/\text{min}$. How fast is the surface area increasing when the length of an edge is $30\text{cm}$?

\newpage

\question
Let $f(x) = \sqrt[3]{1 + 3x}$.
	
	\begin{parts}
	\part[5]
	Find the linearization of $f(x)$ at $a = 0$.
	\vfill
	
	\part[5]
	Use the linearization to approximate the value of $\sqrt[3]{1.03}$.
	\vfill
	
	\end{parts}
	
\newpage

\question
Let $f(x) = \displaystyle \frac{x}{1 - x^2}$. 

	\begin{parts}
	
	\part[4]
	Using \textbf{Calculus}, find the vertical asymptotes (if any) and horizontal asymptotes (if any) of the function $f(x)$.
	
	\vfill
	
	\part[4]
	The first derivative of $f$ is $f'(x) = \frac{1 + x^2}{(x^2 -1)^2}$. Find the critical numbers (if any) and the interval(s) of increase and decrease.
	
	\vfill
	
	\newpage
	
	...\textit{Question 3 continued}...
	
	\part[4] 
	The second derivative of $f$ is $f''(x) = -\frac{2x (3 + x^2)}{(x^2 - 1 )^3}$. Find the $x$-coordinate of the inflection points (if any) and the interval(s) of concavity.
	
	\vfill
	
	\part[4]
	Using one of the derivative tests, find the local maximum(s) and/or local minimum(s) of the function.
	
	\vfill
	
	\newpage
	
	...\textit{Question 3 continued}...
	
	\part[4]
	Sketch the graph of the function $f$ in the axes below.
	
	\begin{center}
		\begin{tikzpicture}
		\draw[black, very thick, ->, >=latex] (-7.6, 0) -- (7.6, 0)node[right]{{\large $x$}};
		\draw[black, very thick, ->, >=latex] (0, -5) -- (0, 9.6)node[above]{{\large $y$}};
		\foreach \x in {-2, -1,  1,  2}{%
			\draw[black, very thick] ({3.5 * \x}, 0.1) -- ({3.5 * \x} , -0.1)node[below]{$\x$};}
		\foreach \y in { -1, 1, 2, 3}{%
			\draw[black, very thick] (0.1, {3*\y}) -- (-0.1, {3*\y})node[left]{$\y$};}
		\end{tikzpicture}
	\end{center}
	%\includegraphics[scale=0.5]{graphEmpty.png}
	
	\end{parts}
	
\newpage

\question 

Compute the following limits. If the limit does not exist, write explicitly DNE. Make sure to describe the method(s) used to obtain the value of the limit.

	\begin{parts}
	\part[5]
	$\displaystyle \lim_{x \ra \infty} \frac{3x^4 + x - 5}{6x^4 - 2x^2 + 1}$.
	\vfill
	
	\part[5]
	$\displaystyle \lim_{x \ra -\infty} \frac{\sqrt{4x^2 + 1}}{3x - 1}$.
	
	\vfill
	\end{parts}
	
\newpage

\question[10]
Find two positive integers such that the sum of the first number and four times the second number is 1000 and the product of the numbers is as large as possible.

\newpage

\question
Answer the following questions.
	\begin{parts}
	\part[5]
	Find the most general antiderivative of $f(x) = 4 \sqrt{x} + \cos x - 2 \sec^2 x$.
	\vfill
	
	\part[5]
	Find $f(x)$ if $f''(x) = 1 - 6x + 48x^2$, $f(0) = 1$ and $f'(0) = 2$.
	\vfill
	
	\part[5]
	Find $f(t)$ if $f' (t) = \displaystyle \frac{t^2 + \sqrt{t}}{t}$ and $f (1) = 3$.
	\vfill
	\end{parts}
	
\newpage

\question
The graph of the function $f(x)= \frac{1}{1 + x^2}$ is given below.

\begin{center}
\includegraphics[scale=0.382]{pic-Q7.png}
\end{center}

\begin{parts}
\part[5]
Estimate the area bounded by the graph of $f(x)$ and the $x$-axis from $a = 0$ to $b = 2$ using two rectangles and right endpoints rule. Is your answer over or under estimating the actual area?
\vfill

\part[5]
Estime the area bounded by the graph of $f(x)$ and the $x$-axis from $a = -2$ to $b = 0$ using two rectangles and the right endpoints rule. Is your answer over or under estimating the actual area?
\vfill
\end{parts}

\newpage

\question
Answer the following questions.

\begin{parts}
\part[5]
Sketch the region whose area is equal to
	\begin{align*}
	\lim_{n \ra \infty} \sum_{i = 1}^n \frac{\pi}{n} \sin \big( \frac{i \pi}{n} \big) .
	\end{align*}
\vfill

\part[5]
Find the number $c$ satisfying the Mean-Value Theorem with $f(x) = x^2$ on $[0, 2]$.
	
\vfill

\newpage
	
\part[5]
Let $x_1 = -1$. Use Newton's method to find the second approximation $x_2$ to the root of the equation
	\begin{align*}
	2x^3 - 3x^2 + 2 = 0 .
	\end{align*}
	
\end{parts}

\end{questions}

\newpage

\textbf{Do not write on this page.}

\vfill



\vspace*{1cm}

\textit{For official use only:}
\begin{center}
\gradetable[h][questions]
\end{center}

\vspace*{1cm}

\end{document}