\documentclass[addpoints, 12pt]{exam}%, answers]
\usepackage[utf8]{inputenc}
\usepackage[T1]{fontenc}

\usepackage{lmodern}
\usepackage{arydshln}
\usepackage[margin=2cm]{geometry}

\usepackage{enumitem}

\usepackage{amsmath, amsthm, amsfonts, amssymb}
\usepackage{graphicx}
\usepackage{tikz}
\usetikzlibrary{arrows,calc,patterns}
\usepackage{pgfplots}
\pgfplotsset{compat=newest}
\usepackage{url}
\usepackage{multicol}
\usepackage{thmtools}

\usepackage{caption}
\usepackage{subcaption}

\usepackage{pifont}

% MATH commands
\newcommand{\bC}{\mathbb{C}}
\newcommand{\bR}{\mathbb{R}}
\newcommand{\bN}{\mathbb{N}}
\newcommand{\bZ}{\mathbb{Z}}
\newcommand{\bT}{\mathbb{T}}
\newcommand{\bD}{\mathbb{D}}

\newcommand{\cL}{\mathcal{L}}
\newcommand{\cM}{\mathcal{M}}
\newcommand{\cP}{\mathcal{P}}
\newcommand{\cH}{\mathcal{H}}
\newcommand{\cB}{\mathcal{B}}
\newcommand{\cK}{\mathcal{K}}
\newcommand{\cJ}{\mathcal{J}}
\newcommand{\cU}{\mathcal{U}}
\newcommand{\cO}{\mathcal{O}}
\newcommand{\cA}{\mathcal{A}}
\newcommand{\cC}{\mathcal{C}}
\newcommand{\cF}{\mathcal{F}}

\newcommand{\fK}{\mathfrak{K}}
\newcommand{\fM}{\mathfrak{M}}

\newcommand{\ga}{\left\langle}
\newcommand{\da}{\right\rangle}
\newcommand{\oa}{\left\lbrace}
\newcommand{\fa}{\right\rbrace}
\newcommand{\oc}{\left[}
\newcommand{\fc}{\right]}
\newcommand{\op}{\left(}
\newcommand{\fp}{\right)}

\newcommand{\ra}{\rightarrow}
\newcommand{\Ra}{\Rightarrow}

\renewcommand{\Re}{\mathrm{Re}\,}
\renewcommand{\Im}{\mathrm{Im}\,}
\newcommand{\Arg}{\mathrm{Arg}\,}
\newcommand{\Arctan}{\mathrm{Arctan}\,}
\newcommand{\sech}{\mathrm{sech}\,}
\newcommand{\csch}{\mathrm{csch}\,}
\newcommand{\Log}{\mathrm{Log}\,}
\newcommand{\cis}{\mathrm{cis}\,}

\newcommand{\ran}{\mathrm{ran}\,}
\newcommand{\bi}{\mathbf{i}}
\newcommand{\Sp}{\mathrm{span}\,}
\newcommand{\Inv}{\mathrm{Inv}\,}
\newcommand\smallO{
  \mathchoice
    {{\scriptstyle\mathcal{O}}}% \displaystyle
    {{\scriptstyle\mathcal{O}}}% \textstyle
    {{\scriptscriptstyle\mathcal{O}}}% \scriptstyle
    {\scalebox{.7}{$\scriptscriptstyle\mathcal{O}$}}%\scriptscriptstyle
  }
\newcommand{\HOL}{\mathrm{Hol}}
\newcommand{\cl}{\mathrm{clos}}
\newcommand{\ve}{\varepsilon}

\DeclareMathOperator{\dom}{dom}

%%%%%% Définitions Theorems and al.
%\declaretheoremstyle[preheadhook = {\vskip0.2cm}, mdframed = {linewidth = 2pt, backgroundcolor = yellow}]{myThmstyle}
%\declaretheoremstyle[preheadhook = {\vskip0.2cm}, postfoothook = {\vskip0.2cm}, mdframed = {linewidth = 1.5pt, backgroundcolor=green}]{myDefstyle}
%\declaretheoremstyle[bodyfont = \normalfont , spaceabove = 0.1cm , spacebelow = 0.25cm, qed = $\blacktriangle$]{myRemstyle}

%\declaretheorem[ style = myThmstyle, name=Th\'eor\`eme]{theorem}
%\declaretheorem[style =myThmstyle, name=Proposition]{proposition}
%\declaretheorem[style = myThmstyle, name = Corollaire]{corollary}
%\declaretheorem[style = myThmstyle, name = Lemme]{lemma}
%\declaretheorem[style = myThmstyle, name = Conjecture]{conjecture}

%\declaretheorem[style = myDefstyle, name = D\'efinition]{definition}

%\declaretheorem[style = myRemstyle, name = Remarque]{remark}
%\declaretheorem[style = myRemstyle, name = Remarques]{remarks}

\newtheorem{theorem}{Théorème}
\newtheorem{corollary}{Corollaire}
\newtheorem{lemma}{Lemme}
\newtheorem{proposition}{Proposition}
\newtheorem{conjecture}{Conjecture}

\theoremstyle{definition}

\newtheorem{definition}{Définition}[section]
\newtheorem{example}{Exemple}[section]
\newtheorem{remark}{\textcolor{red}{Remarque}}[section]
\newtheorem{exer}{\textbf{Exercice}}[section]


\tikzstyle{myboxT} = [draw=black, fill=black!0,line width = 1pt,
    rectangle, rounded corners = 0pt, inner sep=8pt, inner ysep=8pt]

\begin{document}
	\noindent \hrulefill \\
	MATH-241 \hfill Created by Pierre-O. Paris{\'e}\\
	Worksheet 05 \hfill Fall 2022\\\vspace*{-0.7cm}
	
	\noindent\hrulefill
	
\vspace*{0.5cm}

\noindent\makebox[\textwidth]{\textbf{Last name:}\enspace \hrulefill}
\makebox[\textwidth]{\textbf{First name:}\enspace\hrulefill}
\makebox[\textwidth]{\textbf{Section:}\enspace\hrulefill}

\vspace*{0.25cm}
\begin{center}
\gradetable[h][questions]
\end{center}
\vspace*{0.25cm}

{\bf Instructions:} You must answer all the questions below and give your solutions to the TA at the end of the recitation. Write your solutions directly on the worksheet. Late worksheet will not be accepted.

\qformat{\rule{0.3\textwidth}{.4pt} \begin{large}{\textsc{Question}} \thequestion \end{large} \hspace*{0.2cm} \hrulefill \hspace*{0.1cm} \textbf{(\totalpoints\hspace*{0.1cm} pts)}}

\vspace*{0.5cm}

\begin{questions}

\question

Compute the derivative of the following functions:
	\begin{parts}
	\part[5]
	$f(x) = x^2 \sec (x)$.
	\begin{solution}
	{\color{red}
	Using the product rule, we have
		\begin{align*}
		f' (x) = (x^2)' \sec (x) + x^2 (\sec (x))' &= 2x \sec (x) + x^2 \sec (x) \tan (x) \\
		&= x \sec (x) \op 2 + x \tan (x) \fp .
		\end{align*}
	}
	\end{solution}
	
	\part[5]
	$f(x) = \displaystyle \frac{\sqrt{x^2 + x}}{\sin (x)}$.
	\begin{solution}
	{\color{red}
	Using the quotient rule, we have
		\begin{align*}
		f' (x) = \frac{(\sqrt{x^2+ x})' \sin (x) - \sqrt{x^2 + x} (\sin (x))'}{\sin^2 (x)} .
		\end{align*}
	Now, we have
		\begin{align*}
		\frac{d}{dx} \op \sqrt{x^2 + x} \fp = \frac{1}{2 \sqrt{x^2 + x}} \frac{d}{dx} \op x^2 + x \fp = \frac{1}{2 \sqrt{x^2 + x}} (2x + 1) = \frac{2x + 1}{2 \sqrt{x^2 + x}} .
		\end{align*}
	Therefore, we get
		\begin{align*}
		f' (x) &= \frac{\frac{2x + 1}{2 \sqrt{x^2 + x}} \sin (x) - \cos (x) \sqrt{x^2 + x}}{\sin^2 (x)} \\
		&= \frac{ (2x + 1) \sin (x) - 2 (x^2 + x) \cos (x)}{2 \sqrt{x^2 + x} \sin^2 (x)} .
		\end{align*}
	}
	\end{solution}
	\end{parts}
	
\newpage

\phantom{2}

\newpage

\question[10]

Isolate $y$ and find an expression for $y'$ if
	$$
	x^2 + y^2 = 1 .
	$$
	\begin{solution}
	{\color{red}
	We have
		\begin{align*}
		y^2 = 1 - x^2 \Ra y = \pm \sqrt{1 - x^2} .
		\end{align*}
	For the plus sign, we have
		\begin{align*}
		y' = \frac{d}{dx} \op \sqrt{1 - x^2} \fp = \frac{1}{2 \sqrt{1 - x^2}} \frac{d}{dx} \op 1 - x^2 \fp = \frac{1}{2 \sqrt{1 - x^2}} \op -2x \fp = - \frac{x}{\sqrt{1 - x^2}} .
		\end{align*}
	For the minus sign, we have
		\begin{align*}
		y' = \frac{d}{dx} \op - \sqrt{1 - x^2} \fp = - \frac{1}{2 \sqrt{1 - x^2}} \frac{d}{dx} \op 1 - x^2 \fp = - \frac{1}{2 \sqrt{1 - x^2}} (-2x) = \frac{x}{\sqrt{1 -x^2}} .
		\end{align*}
	}
	\end{solution}

\newpage

\phantom{2}

\end{questions}

\end{document}