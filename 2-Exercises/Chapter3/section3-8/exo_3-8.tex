\include{../../semesterTerm.tex}

\begin{document}
	\noindent \hrulefill \\
	MATH-241 \hfill Pierre-Olivier Paris{\'e}\\
	Solutions Section 3-8 \hfill \semester \\\vspace*{-1cm}
	
	\noindent\hrulefill
	
	\spc
	
	\exo{6}
	\\
	We set $f (x) = 2x^3 - 3x^2 + 2$. The derivative of the function is
		\begin{align*}
		f'(x) = 6x^2 - 6x .
		\end{align*}		
	Newton's algorithm gives
		\begin{align*}
		x_{n+1} = x_n - \frac{f (x_n)}{f'(x_n)} .
		\end{align*}			
		
	We start we $x_1 = -1$. We have
		\begin{align*}
		f(x_1 ) = -3 \quad \text{ and } \quad f'(x_1) =  12 .
		\end{align*}	
	Therefore, we obtain
		\begin{align*}
		x_2 = -1 + \frac{3}{12} = -1 + \frac{1}{4} = -\frac{3}{4} .
		\end{align*}
		
	We now use $x_2$ to obtain $x_3$. We have
		\begin{align*}
		f(x_2) = -0.53125 \quad \text{ and } \quad f'(x_2) =  63/8 .
		\end{align*}
	Therefore, we obtain
		\begin{align*}
		x_3 = -\frac{3}{4} + \frac{0.53125}{63/8} \approx -0.6825
		\end{align*}				

	\spc
	
	\exo{34}
	\\
	To find the maximum value, we will use the interval method. We have to find the critical numbers inside $(0, \pi )$. The derivative of $f$ is
		\begin{align*}
		f'(x) = \cos x - x \sin x .
		\end{align*}
	The derivative exists everywhere, so the critical numbers are the zero of $f'$. We will use Newton's method to find the zero of $f'$. If $x = 0$, then $f'(x) = 0$. But $x$ is not inside the interval $(0, \pi )$. We will search for another zero inside $(0, \pi )$. The Newton's method tells us that the critical number $c$ will be approximated by
		\begin{align*}
		x_{n-1} - f'(x_{n-1})/f''(x_{n-1})
		\end{align*}
	where $x_1$ is an initial guess within $(0, \pi )$. 
	
	Let $x_1 = \pi/2$. We have $f''(x) = -2\sin x - x\cos x$. So
		\begin{align*}
		c \approx x_{n-1} - \frac{\cos x_{n-1} - x_{n-1} \sin (x_{n-1})}{-2 \sin x_{n-1} - x_{n-1} \cos x_{n-1}} .
		\end{align*}
	Apprying Newton's method several times, we get the following approximations of $c$:
	\begin{center}
	\begin{tabular}{c|c}
	Iteration & $x_n$ \\\hline
	$2$ & 0.7853981633974483 \\
	$3$ & 0.8624434632122491 \\
	$4$ & 0.8603349794247831 \\
	$5$ & 0.8603335890199867 \\
	$6$ & 0.8603335890193797
	\end{tabular}
	\end{center}
We see that, after the fifth iteration, the first six digits are stable. So $c \approx 0.860333$.

Now, we have
	\begin{align*}
	\max f(x) = \max \{ f(0) , f(0.860333) , f(\pi ) \} = \max \{ 0 , 0.561096 , -3.141592 \} = 0.561096 .
	\end{align*}
	
\end{document}