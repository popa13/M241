\include{../../semesterTerm.tex}

\begin{document}
	\noindent \hrulefill \\
	MATH-241 \hfill Pierre-Olivier Paris{\'e}\\
	Solutions Section 2-7 \hfill \semester \\\vspace*{-1cm}
	
	\noindent\hrulefill
	
	\spc
	
	\exo{6(a)}
	\\
	The slope of the tangent line from $0$ to $1$ is decreasing. Therefore, the object is slowing down between $0$ and $1$. 
	
	At $1$, it stoped because the slope is zero. 
	
	Between $1$ and $0$, the object is speeding up because the slope is getting more negative. 
	
	Then between $2$ and $3$, it is slowing down because the slope is getting closer and closer to $0$. 
	
	At $3$, the slope is zero and therefore the object has stoped.
	
	Finally, the object is speeding up when the time is greater than $3$. 
	
	\spc
	
	\exo{20(a)}
	\\
	Using the power rule, we find that
		\begin{align*}
		\frac{dF}{dr} = \frac{-2 GmM}{r^3} .
		\end{align*}
	The derivative means the rate of change of the force each time away of the center of the body. The minus sign means that the force diminishes when we get further and further away from the object.
	
\end{document}